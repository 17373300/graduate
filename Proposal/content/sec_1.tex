\section{论文研究背景与意义}

\subsection{论文选题背景}
参考文献引用如下\upcite{yan2018spatial,wu2020comprehensive,hu2015,jdmsugg,lovell1999development}

\textbf{\color{red}(可以从如下几个方面进行论述:1、学术界理论研究背景,2、项目研究背景,3、实际应用背景)}
\subsubsection{学术界理论研究背景}

\subsubsection{项目研究背景}

\subsubsection{实际应用背景}



\subsection{研究现状概述}

针对本论文遇到的问题,XXX等人的方法存在XXXX问题。
\textbf{\color{red}
(用2页左右的篇幅,对文献综述所罗列的研究现状进行总结和分析,并列举与论文密切相关的几项工作)}



\subsection{研究目标与创新性}
本论文工作拟完成以下几个研究目标:

首先,针对当前司法预测工作
并没有针对案情描述的文本内容与法律条文的维度关联的相关研究的不足,
本工作拟提出一个案情描述文本与法条维度匹配的数据集,
并将在学术界公开。
此外将提出一个案情描述文本重要性预测模型,
实现法律条文与案情描述文本段的对应关系预测。
在此之后,也将基于从案情描述中挖掘的判案论点和抽取的关键信息,
实现一些常见司法预测任务,例如罪名预测等。

其次,针对司法证据预测这个课题现有研究的不足,
本工作拟提出一个司法证据数据集,
并将在学术界公开。
此外将提出一个司法证据预测模型,
实现针对案情描述文本可解释地预测其所需的完备证据集。

最后,将上述的模型组合成完整的司法辅助系统,
开发的软件系统将用于辅助司法从业者以及其他相关人员
分析理解案情信息并做出有关决策。

论文将为智能司法工作,
引入了案情描述文本分析任务和证据预测任务,
提高了其他司法任务的可解释性,
为司法从业者提供了额外的信息来辅助理解和判决案情。

% \textbf{\color{red}(描述论文的目标以及成果,
% 目标是解决什么问题/探索新的方向,成果可以是以下几种形式:
% 1发表论文、2申请专利、3获得软件著作权、4开发装置、5开发软件模块或者系统、6构建一个数据集)}
% 针对XXX领域的不足,研究XXXX方法/开发XXX系统,解决XXX问题或者:
% 探索XXX领域某方面的新思路。研究成果计划发表于XXX会议、申请XXX项发明专利、
% 获得XXX项软件著作权、开发的软件系统将用于XXX、构建的数据集将在学术界公开。
% \textbf{\color{red}(描述论文工作的创新性,与现有研究和工程方案的区别,
% 侧重于理论方法研究的论文可以写方法思路的创新性,侧重于工程实践的论文,
% 可以写系统方案、解决问题的新思路)}
% 论文将引入XXX思路、改进XXX方法、探索XXX理论,从而提高XXX准确率,实现XXX效果、解决XXX问题。
