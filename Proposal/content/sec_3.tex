\section{论文工作安排计划}

\subsection{工作进度安排}

\begin{table}[h]
    \centering
    \begin{tabular}{|c|c|}
        \hline
        \textbf{日期}   & \textbf{计划与安排}                                 \\
        \hline
        2022.12-2023.03 & \makecell[c]{标注、整理、构建数据集,阅读相关论文, \\根据现有方法和模型构建基线模型。}\\
        \hline
        2023.04-2023.08 & 设计并实现模型,完成对应实验,分析实验结果。        \\
        \hline
        2023.08-2023.10 & 设计实现可视化展示部分,构建完整系统。              \\
        \hline
        2023.10-2023.11 & 整理总结研究成果,撰写毕业论文。                    \\
        \hline
    \end{tabular}
    \caption{工作进度安排}
    \label{plan1}
\end{table}

本论文的工作进度安排如\cref{plan1}所示。
\subsection{论文工作基础}
\textbf{\color{red}(在以下几个方面选择几个方面进行说明:1)收集或者准备的数据集、2)完成或者正在进行的调研工作;3)已完成或者正在进行的理论推导、4)已经完成或者正在进行的开发系统或软件模块、5)正在进行或者完成的实验与实验结果)}

\subsubsection{数据集}
实验目前所用数据集,来自三个方面。
\begin{enumerate}
    \item
          本工作构建了一个刑事判决文书的证据数据集。
          证据数据集中,每份判决书由案情描述、证据集、
          罪名、相关法条、判决结果构成。
          数据集来自于中国裁判文书网【引用】公开的
          原始刑事一审判决文书。
          由于判决文书的书写规范性、严谨性,
          本工作通过一定的规则,
          从原始文书中抽取出所需的信息,
          从而构建证据数据集
          (之后会考虑聘请具有法律背景的专家来进行人工标注)。
    \item
          本工作构建了一个案情描述关键信息数据集。
          在自动抽取构建的证据数据集中,
          本工作随机选择了一部分的判决文书,
          人工标注了基于法条指导下的案情描述中的关键信息。
          具体来说,
          基于法条解耦出的维度,
          本工作标注了每一个维度对应的案情描述中的语句,
          本工作认为这一部分的信息是案情中的“关键信息”,
          并且作为了任务预测目标。
    \item
          本工作使用了司法决策任务相关论文【引用】常用的数据集,
          来自法研杯2018年的CAIL数据集\upcite{xiao2018cail2018}。
          CAIL数据集中的每份判决书由案情描述
          和判决结果(适用法律条文、罪名、刑期)组成。
          CAIL有两个子数据集:CAIL-big和CAILsmall。
          本工作按照现有工作\upcite{xu2020distinguish,yang2019legal}对CAIL数据集进行预处理。
          具体而言,
          论文过滤掉了具有多个适用罪名和法律条文的案例样本。
          此外,本工作仅保留不少于 100 个相应案例样本的罪名和法律条文,
          并且删除了所有的二审案例样本。

\end{enumerate}
\subsubsection{调研工作}


\subsubsection{现有实验结果}


% 1)收集或者准备的数据集、
% 2)完成或者正在进行的调研工作、
% 3)已完成或者正在进行的理论推导、
% 4)已经完成或者正在进行的开发系统或软件模块、
% 5)正在进行或者完成的实验与实验结果
\subsection{可能遇到的问题以及解决途径}
本文所设涉及的工作的主要的技术难点有:
\begin{enumerate}
    \item
          对于司法文书的案情描述,
          其长度并不固定,
          有较多案情描述的长度超出BERT模型所能接纳的最大长度,
          这使得直接使用BERT模型来理解案情描述的语义信息产生问题。
          所以,本工作所需要处理长文本建模这一技术难点。
          本工作拟对案情描述进行分段的语义处理的方式,
          综合多段的语义结合成长文本的案情描述语义信息。

    \item
          在案情描述的关键信息抽取中,
          无论是基于强化学习还是注意力机制的机理解释,
          其结果的连贯性和可读性都是重要的问题。
          关键信息抽取若以词为单位,
          抽取出的关键事实很可能是不可读的词语碎片,
          而若基于句子,则粒度过于粗糙。
          因此如何构造关键信息数据集,
          并且基于此进行文本分类实验,
          提高模型给定关键信息的可读性,
          会在之后的实验中逐步探索。

    \item
          当从案情描述中总结出判案论点和关键信息之后,
          如何使用这一部分的语义信息来指导后续的司法任务
          以及证据预测任务,
          并且体现可解释性是难点之一。
          在本工作中,
          其指导和可解释证据预测体现在判案论点和证据集的对应关系上,
          其指导和可解释其他司法任务体现在,
          本工作使用抽取的部分关键信息和挖掘的判案论点,
          替代原有的案情描述全文,交给下游司法任务,
          由此论证。
          但是如何明确对应关系,
          以及只使用关键信息,
          而非全文信息导致了一定量的信息丢失,
          是否能给下游司法任务带来提升,
          都是本工作在开展实验的过程中需要继续面对的问题。

    \item
          虽然司法判决文书的写作具有规范性和严谨性,
          但是由于全国不同地域不同级别的法院对法律文书的书写习惯不同,
          司法判决文书的行文结构会存在一定的差异。
          本工作基于规则从判决文书中抽取出的证据集会参杂部分噪声,
          从而影响证据集的聚类以及预测工作。
          因此,本工作通过方法抽取出证据后,
          仍需要大量的数据清洗工作和部分人工过滤与处理得到较高质量的数据集。

    \item
          在证据生成的实验中,
          可能会存在某一个案情描述片段没用对应任何证据,
          或者某一个案情描述片段对应了多个证据,
          但是判决书中罗列的证据彼此之间没有先后关系。
          因此,这种情况对序列到序列模型提出了更高的要求,
          在之后的模型构建中,
          会针对这种特殊的场景设计与之相对应的模型结构。


\end{enumerate}

\textbf{\color{red}
    (描述论文可能遇到理论证明、工程开发、核心算法不足、实验数据等方面困难以及应对措施)}